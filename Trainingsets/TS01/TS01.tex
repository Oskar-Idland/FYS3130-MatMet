\documentclass{article}
\usepackage{amsmath}
\usepackage{titlesec}
\usepackage[mathletters]{ucs}
\usepackage[utf8x]{inputenc}
\usepackage[margin=1.5in]{geometry}
\usepackage{enumerate}
\newtheorem{theorem}{Theorem}
\usepackage[dvipsnames]{xcolor}
\usepackage{pgfplots}
\pgfplotsset{compat=1.18}
\setlength{\parindent}{0cm}
\usepackage{graphics}
\usepackage{graphicx} % Required for including images
\usepackage{subcaption}
\usepackage{bigintcalc}
\usepackage{pythonhighlight} %for pythonkode \begin{python}   \end{python}
\usepackage{appendix}
\usepackage{arydshln}
\usepackage{physics}
\usepackage{booktabs} 
\usepackage{adjustbox}
\usepackage{mdframed}
\usepackage{relsize}
\usepackage{physics}
\usepackage[thinc]{esdiff}
\usepackage{esint}  %for lukket-linje-integral
\usepackage{xfrac} %for sfrac
\usepackage{hyperref} %for linker, må ha med hypersetup
\usepackage[noabbrev, nameinlink]{cleveref} % to be loaded after hyperref
\usepackage{amssymb} %\mathbb{R} for reelle tall, \mathcal{B} for "matte"-font
\usepackage{listings} %for kode/lstlisting
\usepackage{verbatim}
\usepackage{graphicx,wrapfig,lipsum,caption} %for wrapping av bilder
\usepackage{mathtools} %for \abs{x}
\usepackage[english]{babel}
\usepackage{cancel}
\definecolor{codegreen}{rgb}{0,0.6,0}
\definecolor{codegray}{rgb}{0.5,0.5,0.5}
\definecolor{codepurple}{rgb}{0.58,0,0.82}
\definecolor{backcolour}{rgb}{0.95,0.95,0.92}
\lstdefinestyle{mystyle}{
    backgroundcolor=\color{backcolour},   
    commentstyle=\color{codegreen},
    keywordstyle=\color{magenta},
    numberstyle=\tiny\color{codegray},
    stringstyle=\color{codepurple},
    basicstyle=\ttfamily\footnotesize,
    breakatwhitespace=false,         
    breaklines=true,                 
    captionpos=b,                    
    keepspaces=true,                 
    numbers=left,                    
    numbersep=5pt,                  
    showspaces=false,                
    showstringspaces=false,
    showtabs=false,                  
    tabsize=2
}

\lstset{style=mystyle}
\author{Oskar Idland}
\title{Training Set 1}
\date{}
\begin{document}
\maketitle
%\tableofcontents
\newpage

\section*{Problem T1.1}
\subsection*{a)}
\[
x = r \cos θ \quad y = r \sin θ \quad
\]
\[
r = \sqrt{2} \quad , \quad  θ = \frac{5π}{4}
\]
\[
x = \sqrt{2} \cos \frac{5π}{4} = -1 \quad , \quad y = \sqrt{2} \sin \frac{5π}{4} = -1
\]
\[
\underline{\underline{z = -1 - i}}
\]
 
\subsection*{b)}
\[
r = \sqrt{x^2 + y^2} \quad , \quad θ = \arctan \frac{y}{x}
\]
\[
r = \sqrt{3 + 1} = 2 \quad , \quad  θ = \arctan \frac{1}{\sqrt{3}} = -\frac{π}{6}
\]
\[
\frac{(1+i)^{48}}{(\sqrt{3}-i)^{25}} = \frac{\left(\sqrt{2}e^{iπ/4}\right)^{48}}{\left(2e^{-iπ / 6}\right)^{25}} = 2^{24 - 25} e^{iπ(12 + 25 / 6)} = \frac{1}{2} e^{iπ(12 + 25 / 6)} 
\]
\[
x = \frac{1}{2} \cos (π(12 + 25 / 6))
\]
\[
y = \frac{1}{2} \sin (π(12 + 25 / 6))
\]
\[
\underline{\underline{z = \frac{1}{2} \cos (π(12 + 25 / 6)) + i \frac{1}{2} \sin (π(12 + 25 / 6))}}
\]

\subsection*{c)}
\[
\frac{e^{1+3πi}}{e^{-1 + iπ / 2}} = e^{1 -(-1) + 3πi - iπ / 2} = e^{2 + 5πi / 2} = e^2 e^{i5π / 2} = e^2 e^{iπ / 2}
\]
\[
x = 0 \quad , \quad  y = 1 ⇒ \underline{\underline{z = ie^2}}
\]

\subsection*{d)}
\[
x = -8 \quad , \quad  y = 8\sqrt{3}
\]
\[
r = \sqrt{64 + 192} = 16 \quad , \quad  θ = \arctan \frac{\sqrt{3}}{-1} = \frac{2π}{3}
\]
\[
z = (8i\sqrt{3} - 8)^{1 / 4} = \left(16 e^{i2π / 3}\right)^{1 / 4} = 2e^{iπ / 6} 
\]
\[
w_0 = 2e^{iπ / 6} \quad , \quad   w_1 = 2e^{i(π / 6 + 2π / 4)} = 2e^{i(π / 6 + π / 2)} = 2e^{i 2π / 3}
\]
\[
w_2 = 2e^{iπ(1 / 6 + 2/2)} = 2e^{iπ 7 / 6} \quad , \quad  w_3 = 2e^{iπ (1 / 6 + 3 / 2)} = 2e^{iπ 5 / 3}
\]

\subsection*{e)}
\[
z = 8 = 8e^{i0} 
\]
\[
z^{1 / 3} = w_0 = 8^{1 / 3} = 2 \quad , w_1 = 2e^{i2π / 3} \quad , w_2 = 2e^{-i2π / 3}
\]
\[
∑_{i=0}^{2} w_i = 2 + 2e^{i2π / 3} + 2e^{-i2π / 3} = 2 + 2\left(\cos \frac{2π}{3} + i \sin \frac{2π}{3}\right) + 2\left(\cos \frac{2π}{3} - i \sin \frac{2π}{3}\right) 
\]
\[
2 + 4 \cos 2π / 3 = 2 + 4 \left(-\frac{1}{2}\right) = 0
\]
After finding one root we rotate around the unit circle by $2π / n$. If $n$ is even then for each root with argument $θ$ there is another root with argument $-θ$. As the angles cancel out, the sum of the numbers below the real axis will be some negative number $-a$, and the numbers above will be $a$. The sum of all the roots will then be $0$. 


\subsection*{f)}
\[
z = i = e^{iπ / 2}
\]
\[
z^{1 / 3} = w_0 = e^{iπ / 6}, \quad w_1 = e^{iπ (1 / 6 + 2 / 3)} = e^{iπ 5 / 6}, \quad w_2 = e^{iπ (1 / 6 + 4 / 3)} = e^{- iπ 3 / 2}
\]

\subsection*{g)}
\[
z = 2 + 2\sqrt{3}i = 4e^{iπ / 3}
\]
\[
w = \sqrt{z} = 2e^{iπ / 6}
\]
\[
\sqrt{w} = w_0 = \sqrt{2}e^{i π / 12} \quad , w_1 = \sqrt{2}e^{i 13π / 12} = \sqrt{2}e^{- i 11π / 12}
\]
\[
w_0 = \sqrt{2}e^{i π / 12} = \sqrt{2} \left(\cos π / 12 + i\sin π / 12\right)
\]
\[
w_1 = \sqrt{2}e^{-i 11π / 12} = \sqrt{2} \left(\cos (-11π / 12) + i \sin \left(-11 π / 12\right)\right)
\]
Cosine values at 180 degrees are each other negative. Sine values at 180 degrees are also each other negative. 
\[
w_0 + w_1 = \sqrt{2} \left(0.966 - 0.966 +  i(0.259 - 0.259)\right) = 0
\]

\section*{Problem T1.2}
\subsection*{a)}
Applying the convergence test:
\[
\lim_{n \to ∞} \left|\frac{a_{n+1}}{n}\right| < 1
\]
\[
\lim_{n \to ∞} \left|\frac{(n+1)(n+2)(z-2i)^{n+1}}{n(n+1)(z-2i)^{n}}\right| < 1
\]
\[
\lim_{n \to ∞} \left|\frac{(n+2) (z - 2i)}{n}\right| < 1
\]
$(n+2) / n$ is always positive and converges to $1$. 
\[
\left|z - 2i\right| < 1
\]
$z$ represent a point in the complex plane and $2i$ is an offset from the origin. The inequality represents a circle with radius $1$ centered at $(0, 2i)$

\subsection*{b)}
\[
e^{z} = \sum_{n=0}^{\infty} \frac{z^n}{n!}
\]
We apply the convergence test: 
\[
\lim_{n \to ∞} \left|\frac{a_{n+1}}{a_n}\right| < 1
\]
\[
\lim_{n \to ∞} \left|\frac{n! z^{n+1}}{(n+1)! z^{n}}\right| 
\]
\[
\lim_{n \to ∞} \left|\frac{z}{n}\right| = 0 < 1 
\]
This holds true for all $z$ in the complex plane. 


\section*{Problem T1.3}
\subsection*{a)}
Substituting $\sin z$ with $(e^{iz} - e^{-iz} / 2i)$ and $\cos z$ with $(e^{iz} + e^{-iz} / 2)$:
\[
2 \sin z \cos z = \frac{1}{2} \left((-e^{iz} + e^{-iz})  (e^{iz} + e^{-iz})\right) = \frac{1}{2} \left(-e^{2iz} -1 + 1 + e^{-2iz}\right)
\]
\[
\underline{\underline{\frac{e^{-2iz} - e^{-2iz}}{2} = \sin 2z}}
\]

\subsection*{b)}
Substituting $\sinh z$ with $(e^{z} - e^{-z} / 2)$ and $\cosh z$ with $(e^{z} + e^{-z} / 2)$:
\[
\cosh^2 z - \sinh^2 z = \left(\frac{e^{z} + e^{-z}}{2}\right)^2 - \left(\frac{e^{z} - e^{-z}}{2}\right)^2 = \frac{e^{2z} + 2 + e^{-2z}}{4} - \frac{e^{2z} - 2 + e^{-2z}}{4} = 1
\]

\subsection*{c)}
\[
e^{z} = ∑_{n=0}^{∞} \frac{z^{n}}{n!}
\]
\[
e^{1+iπ} = ∑_{n=0}^{∞} \frac{(1 + iπ)^{n}}{n!}
\]
\[
\underline{\underline{e^{1 + iπ} = e ⋅ -1 = -e}}
\]

\end{document}