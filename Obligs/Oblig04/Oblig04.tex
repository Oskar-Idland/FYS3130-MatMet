\documentclass{article}
\usepackage{amsmath}
\usepackage[mathletters]{ucs}
\usepackage[utf8x]{inputenc}
\usepackage[margin=1.5in]{geometry}
\usepackage{enumerate}
\newtheorem{theorem}{Theorem}
\usepackage[dvipsnames]{xcolor}
\usepackage{pgfplots}
\setlength{\parindent}{0cm}
\usepackage{graphics}
\usepackage{graphicx} % Required for including images
\usepackage{subcaption}
\usepackage{bigintcalc}
\usepackage{pythonhighlight} %for pythonkode \begin{python}   \end{python}
\usepackage{appendix}
\usepackage{arydshln}
\usepackage{physics}
\usepackage{tikz-cd}
\usepackage{booktabs} 
\usepackage{adjustbox}
\usepackage{mdframed}
\usepackage{relsize}
\usepackage{physics}
\usepackage[thinc]{esdiff}
\usepackage{fixltx2e}
\usepackage{esint}  %for lukket-linje-integral
\usepackage{xfrac} %for sfrac
\usepackage[colorlinks=true]{hyperref} %for linker, må ha med hypersetup
\usepackage[noabbrev, nameinlink]{cleveref} % to be loaded after hyperref
\usepackage{amssymb} %\mathbb{R} for reelle tall, \mathcal{B} for "matte"-font
\usepackage{listings} %for kode/lstlisting
\usepackage{verbatim}
\usepackage{graphicx,wrapfig,lipsum,caption} %for wrapping av bilder
\usepackage{mathtools} %for \abs{x}
\usepackage{cancel}
\usepackage[norsk]{babel}
\definecolor{codegreen}{rgb}{0,0.6,0}
\definecolor{codegray}{rgb}{0.5,0.5,0.5}
\definecolor{codepurple}{rgb}{0.58,0,0.82}
\definecolor{backcolour}{rgb}{0.95,0.95,0.92}
\lstdefinestyle{mystyle}{
    backgroundcolor=\color{backcolour},   
    commentstyle=\color{codegreen},
    keywordstyle=\color{magenta},
    numberstyle=\tiny\color{codegray},
    stringstyle=\color{codepurple},
    basicstyle=\ttfamily\footnotesize,
    breakatwhitespace=false,         
    breaklines=true,                 
    captionpos=b,                    
    keepspaces=true,                 
    numbers=left,                    
    numbersep=5pt,                  
    showspaces=false,                
    showstringspaces=false,
    showtabs=false,                  
    tabsize=2
}

\lstset{style=mystyle}
\author{Oskar Idland}
\title{Oblig 4}
\date{}
\begin{document}
\maketitle
\newpage
\section*{Problem 1}
\subsection*{a)}
Finding the Residues as seen in \cref{eq: Residues}:
\begin{equation}\label{eq: Residues}
\underbrace{\operatorname{Res}(f, z_n)}_{b_n} = \lim_{z → z_n} (z-z_n) f(z)
\end{equation}

\[
b_0 = \lim_{z → z_0} (z-z_0)f(z) 
\]
Inserting the value for $z_0$:
\[
b_0 = \lim_{z \to 2} \cancel{(z-2)} \frac{z+2}{\cancel{(z-2)}}
\]
\[
\underline{\underline{b_0= 4}} 
\]

\subsection*{b)}
We rewrite the denominator as follows:
\[
f(z) =  \frac{z+3}{z^3 - 4z^2 - 3z + 18} = \frac{z+3}{(z+2) (z-3)^2}
\]
We can see that we have a residue at $z_0 = 3$ and $z_1 = -2$. Again we find the residue for each singularity. For $z_0$ we must take into account that it is a pole of order 2.:
\[
b_0 = \frac{1}{(2-1)!} \lim_{z → 3} \frac{∂ }{∂ z} \cancel{(z-3)^2} \frac{z+3}{(z+2) \cancel{(z-3)^2}} = \lim_{z → 3} \frac{∂ }{∂ z} \frac{z+3}{(z+2)}  
\]
\[
\frac{∂ }{∂ z} \frac{z+3}{(z+2)} = \frac{(z+2) - (z+3)}{(z+2)^2} = \frac{-1}{(z+2)^2}
\]
\[
b_0 = \lim_{z \to 3} \frac{-1}{(z+2)^2}
\]
\[
\underline{\underline{b_0 = -\frac{1}{25}}}
\]

For $b_1$ we find the residue as normal:
\[
b_1 = \lim_{z → -2} (z+2) \frac{z+3}{(z+2) (z-3)^2} = \lim_{z \to -2} \frac{z+3}{(z-3)^2} = \underline{\underline{\frac{1}{25}}}
\]

\subsection*{c)}
We use 
\[
b_0 = \lim_{z \to 0} (z-0) \sin \left(\frac{1}{z}\right) = \lim_{z \to 0} \ z \sin \left(\frac{1}{z}\right)
\]
For small angles $\sin (z) ≈ z$. We can therefore write:
\[
b_0 = \lim_{z \to 0} \ z \frac{1}{z} = 1
\]


\section*{Problem 2}
\subsection*{a)}
\[
∮_{C} f(z) \mathrm{d}z 
\]
We expand the function:
\[
∮_{C} \left(∑_{0}^{∞} a_n z^{n} + ∑_{-∞}^{0} b_n z^{n}\right)
\]
From the function definition we know the integral evaluates to the following:
\[
∮_{C} z^{n} \mathrm{d}z = \begin{cases}
  0,   &\text{ if } n ≠ -1 \\
  2πi, &\text{ if } n = -1
\end{cases}
\]
\[
∴ \underline{\underline{∮_{C} f(z) \mathrm{d}z = b_1 2πi \operatorname{Res}(f, 0) = b_1 2πi }}
\]

\subsection*{b)}


\section*{Problem 3}
\subsection*{a)}
We factor the denominator:
\[
f(z) = \frac{z+5}{z^2 - z -6} = \frac{z+5}{(z-3)(z+2)}
\]
It's easy to see that we have two simple poles at $z_0 = 3$ and $z_1 = -2$. We can use the Residue Theorem \cref{eq: Residue Theorem} to evaluate the closed curve integral after finding the residues at each singularity:

\begin{equation}\label{eq: Residue Theorem}
∮_{C} f(z) \mathrm{d}z = 2πi ∑_{n}^{} \underbrace{\operatorname{Res}(f, z_n)}_{b_n}
\end{equation}
For $z_0 = 3$: 
\[
b_0 = \lim_{z → 3} (z-3) \frac{z+5}{(z-3)(z+2)} = \lim_{z → 3} \frac{z+5}{z+2} 
\]
\[
\underline{b_0 = \frac{8}{5}}
\]
For $z_1 = -2$:
\[
b_1 = \lim_{z → -2} \cancel{(z+2)} \frac{z+5}{(z-3)\cancel{(z+2)}} = \lim_{z → -2} \frac{z+5}{z-3}
\]
\[
\underline{b_1 = -\frac{3}{5}}
\]
Now we can use the Residue Theorem to evaluate the integral:
\[
\underline{\underline{∮_{C} f(z) \mathrm{d}z = 2πi \left(\frac{8}{5} - \frac{3}{5}\right) = 2πi}}
\]

\subsection*{b)}
Factoring the denominator:
\[
f(z) = \frac{1}{z^3 - 4z^2 - 3z + 18} = \frac{1}{(z+2)(z-3)^2}
\]
We can see that we have a simple pole at $z_0 = -2$ and a pole of order 2 at $z_1 = 3$. 
\[
b_0 = \lim_{z → -2} \cancel{(z+2)} \frac{1}{\cancel{(z+2)}(z-3)^2} = \lim_{z → -2} \frac{1}{(z-3)^2}
\]
\[
\underline{b_0 = \frac{1}{25}}
\]

For $z_1 = 3$ we must take into account that it is a pole of order 2:
\[
b_1 = \frac{1}{(2-1)!} \lim_{z → 3} \frac{∂ }{∂ z} \cancel{(z-3)^2} \frac{1}{(z+2) \cancel{(z-3)^2}} = \lim_{z → 3} \frac{∂ }{∂ z} \frac{1}{(z+2)} = \lim_{z → 3} \frac{-1}{(z+2)^2}
\]
\[
\underline{b_1 = -\frac{1}{25}}
\]

Now we can use the Residue Theorem to evaluate the integral:
\[
\underline{\underline{∮_{C} f(z) \mathrm{d}z = 2πi \left(\frac{1}{25} - \frac{1}{25}\right) = 0}}
\]

\subsection*{c)}
\[
f(z) = \frac{\sin z}{(z - π)^{4}}
\]
This is a pole of order 4 at $z_0 = π$. We calculate the residue accordingly:
\[
b_0 = \frac{1}{(4-1)!} \lim_{z → π} \frac{∂^3 }{∂ z^3} \cancel{(z - π)^{4}} \frac{\sin  z}{\cancel{(z-π)^4}} = -\frac{1}{6} \lim_{z \to π}\cos z
\]
\[
\underline{b_0 = -\frac{1}{6}}
\]
\[
\underline{\underline{∮_{C} f(z) \mathrm{d}z = 2πi \frac{1}{6} = \frac{πi}{3}}}
\]

\subsection*{d)}
\[
f(z) = e^{1 / z}
\]
We have a simple pole at $z_0 = 0$. We expand using the Laurent series:
\[
e^{1 / z} = ∑_{0}^{∞} \frac{1}{n!} \left(\frac{1}{z}\right)^n
\]
The coefficient of $1 / z$ (the first term) is 1 and so the residue is 1. We can now use the Residue Theorem to evaluate the integral:
\[
\underline{\underline{∮_{C} f(z) \mathrm{d}z = 2πi}}
\]
\end{document}